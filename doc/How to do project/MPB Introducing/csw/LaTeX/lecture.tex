\documentclass{beamer}
\usetheme[secheader]{Boadilla}

\usepackage{mathpazo,amsmath,listings}
\definecolor{darkblue}{rgb}{0,0,0.5}
\definecolor{darkgreen}{rgb}{0,0.5,0}
\lstset{language=[LaTeX]TeX,
  basicstyle={\color{darkgreen}},
  morekeywords={%
    part,chapter,subsection,subsubsection,%
    paragraph,subparagraph,%
    appendix}}

\newcommand*{\guide}[1][]{%
  \beamergotobutton{%
    \href{http://www-course.cs.york.ac.uk/csw/LaTeX/example.pdf}{Guide}%
  }#1%
}

\title[\LaTeXe{} Intro]{An Introduction to Type-setting projects in
  \LaTeXe{} with the \lstinline|UoYCSProject| class}

\author[JLJ]{Jeremy Jacob}

\institute[UoY, CS]{The University of York, Department of Computer
  Science}

\date{2009 September 07}

\begin{document}

\section{Title}

\frame{\titlepage}

\section{Introduction}

\begin{frame}
  \frametitle{What is \LaTeX?}

  \LaTeXe{} is a \emph{document description language} built on top of
  Donald Knuth's \TeX{} type-setting engine.

  \vfill

  Cf.\ \texttt{HTML} and \texttt{SGML}/\texttt{XML} applications.
\end{frame}

\begin{frame}[fragile]
  \frametitle{A minimal document}
  
    \begin{tabular}{lcl}
    Source&&Output\\
    \begin{lstlisting}[gobble=6,showspaces=true]
      \documentclass{minimal}
      \begin{document}
      Hello World.
      \end{document}
    \end{lstlisting}
    &&
    \color{darkblue}\rmfamily\tiny
    \framebox[50mm][l]{\begin{tabular}{l}\\Hello World.\\[60mm]\end{tabular}}
  \end{tabular}

\end{frame}

\begin{frame}
  \frametitle{Why use \LaTeXe{}?}
  \begin{itemize}
  \item The sophisticated type-setting algorithm of \TeX, and the
    enhanced algorithm of \emph{pdfe(la)tex}.
    \beamergotobutton{\href{http://www.tug.org/texshowcase/}{the
        \TeX{} showcase}.}
  \item The huge number of pre-defined packages for doing common
    things. \beamergotobutton{\href{http://www.tex.ac.uk/tex-archive/help/Catalogue/}{the
        \TeX{} catalogue}}
  \item The ability to define your own special purpose structures.
  \item Stable basis.
  \item Good for large, academic documents.
  \end{itemize}  
\end{frame}

\begin{frame}
  \frametitle{References}
  There are many good references for \TeX{} and friends.
  
  See ``\emph{A guide to type-setting project reports in \LaTeX{} with
    the \emph{UoYCSproject} class}''. \guide
\end{frame}


\begin{frame}
  \frametitle{UoYCSProject}\framesubtitle{a class for project reports}
  There are many pre-defined document classes:

  \begin{description}[KOMA-Script]
  \item[Base] minimal, article, report, book, letter, slides.
  \item[KOMA-Script] scrartcl, scrreprt, scrbook, scrlttr2.
  \item[Memorandum] memorandum.
  \item[Others] \ldots, beamer, \ldots,UoYCSproject
  \beamergotobutton{\href{http://www-course.cs.york.ac.uk/csw/LaTeX/UoYCSproject.cls}{UoYCSproject}},
  \ldots 
  \end{description}
\end{frame}

\section{The language}
\subsection{General}

\begin{frame}[fragile]
  \frametitle{Text, commands and environments}
  
  A \LaTeXe{} source is a mix of:
  \begin{description}[environments]
  \item[text] \lstinline|Some text.|,
  \item[commands] \lstinline|\LaTeXe|, \lstinline|\frac{2}{3}|, and
  \item[environments]
    \begin{lstlisting}[gobble=6]
      \begin{verse}
        APRIL is the cruellest month, breeding
        \\Lilacs out of the dead land, mixing
        \\...
      \end{verse}
    \end{lstlisting}
  \end{description}
\end{frame}

\begin{frame}[fragile]
  \frametitle{The anatomy of a LaTeX source}
    \begin{lstlisting}[morekeywords={document},gobble=4]
    \documentclass[class options]{class name}
    preamble (definitions and declarations)
    \begin{document} % this is a comment.
    body 
    \end{document}
  \end{lstlisting}
\end{frame}

\subsection{UoYCSproject}

\begin{frame}[fragile]
  \frametitle{The anatomy of a UoYCSproject preamble}
  \begin{lstlisting}[
    basicstyle={\footnotesize\color{darkgreen}},
    gobble=4,
    showspaces=true,
    morekeywords={supervisor,MEng,wordcount,excludes,dedication,abstract,document}]
    \documentclass{UoYCSproject}
    % Order of declarations does not matter.
    \author{Anne Student-Name}
    \title{A Solution to the Problem of $\mathit{P}=\mathit{NP}$}
    \date{30 February 2000}
    \supervisor{Prof. Z. Soporific}
    \MEng
    \wordcount{2,345}\excludes{Appendix~\ref{sec:code}}
    \dedication{To My Cat, Jeoffery}
    \abstract{The well known problem of $P=NP$ is explained,
      together with its significance and a brief history of
      attempts to solve it.  An ingenious solution is presented.}
    \begin{document}
    ...
    \end{document}
  \end{lstlisting}  

  A full list of declarations is given in \guide[, Figure~7.1, P~46].
\end{frame}

\begin{frame}
  \frametitle{Extra definitions and package loading}
  \begin{itemize}
  \item   You can load extra packages and make your own definitions.
  
  \item These go in a file with the same name as your main file, but
    extension `\lstinline|ldf|'. \alert{This is different to the way
      all other classes work.}  (I have implemented
    \lstinline|UoYCSproject| in this way to ensure that packages are
    loaded in the correct order.)
  
  \item Useful packages include: \lstinline|listings|,
    \lstinline|graphics|, \lstinline|graphicx|, \lstinline|pgf/tikz|,
    \lstinline|amsmath|.
  \end{itemize}
\end{frame}

\begin{frame}
  \frametitle{The anatomy of the body}
    \begin{description}[Front matter]
  \item[Front matter] Title pages, abstract, contents, \&c.
  \item[Main matter] The text, divided into (parts,) chapters (,
    sections, subsections, subsubsections, paragraphs and
    subparagraphs).
  \item[Back matter] Bibliography, appendices \&c.
  \end{description}
\end{frame}

\begin{frame}[fragile]
  \frametitle{Front matter}
    \begin{lstlisting}[gobble=4,basicstyle={\small\color{darkgreen}}]
    \maketitle % Compulsory: title pages, table of contents
    \listoffigures % Optional: the list of figures
    \listoftables % Optional: the list of tables
    ... % Optional, package dependent lists,
    ... % e.g. \lstlistoflistings
  \end{lstlisting}
\end{frame}

\begin{frame}[fragile]
  \frametitle{Main matter}
    \begin{lstlisting}[gobble=4]
    \part{title}          % Optional
    \chapter{title}       % Compulsory
    \section{title}       % Optional
    \subsection{title}    % Optional
    \subsubsection{title} % Optional
    \paragraph{title}     % Optional
    \subparagraph{title}  % Optional
    Text.  Text.
  \end{lstlisting}
\end{frame}

\begin{frame}[fragile]
  \frametitle{Back matter}
    \begin{lstlisting}[gobble=4]
    \bibliography{file1,file2} % Construct bibliography
    \appendix % remaining chapters are appendices
    \chapter{title}       % One per appendix
    \section{title}       % Optional
    \subsection{title}    % Optional
    \subsubsection{title} % Optional
    \paragraph{title}     % Optional
    \subparagraph{title}  % Optional
    Text.  Text.
  \end{lstlisting}
\end{frame}

\subsection{General}

\begin{frame}[fragile]
  \frametitle{Text elements}

  \begin{description}[Characters]
  \item[Characters] Can control series, family, shape, colour and size
    of each text character.  See \guide[, \S6.3.3].
  \item[Sentences] \lstinline[showspaces=true]|Sentence one.  Sentence two.|
  \item[Paragraphs]
    \begin{lstlisting}[%
      showspaces=true,gobble=6,basicstyle={\footnotesize\color{darkgreen}}]
      Paragraph one.  %  blank line separates paragraphs

      Paragraph two.
    \end{lstlisting}
  \end{description}
\end{frame}

\begin{frame}[fragile]
  \frametitle{Special features}
  \begin{description}[Context dependent emphasis]
  \item[Context dependent emphasis] \lstinline|\emph{...\emph{...}...}|
  \item[Cross references] Sectional units, floats, equations, \&c.
  \item[Quotations] Short and long
  \item[Citations] 
  \item[Lists] Bulleted, numbered and labelled
  \item[Tables]
  \item[Pictures]
  \item[Floats] Tables, Figures and others.
  \end{description}
\end{frame}

\begin{frame}[fragile]
  \frametitle{Citations and the bibliography}
  \begin{itemize}
  \item Through the \lstinline{natbib} package, set up for IEEE style.
  \item \lstinline|\citep{Joyce:FW}| ---cite parenthesised---
    generates \textcolor{darkblue}{[34]}, assuming
    \lstinline{Joyce:FW} is the label of the 34th reference.

    \alert{Do not use this form as a noun.}
  \item \lstinline|\citet{Joyce:FW}| ---cite as text--- generates
    \textcolor{darkblue}{Joyce~[34]}, assuming also that the author's
    surname is `Joyce'.

    You may use this form as a noun.
  \item Rule: your document should read naturally when all the
    citation markers (numbers in square brackets, plus the brackets)
    are removed.
  \end{itemize}
  Citations are kept in a database in a flat file and processed by a
  program called Bib\TeX{} before inclusion in output
  file. \beamergotobutton{\href{http://www-course.cs.york.ac.uk/csw/LaTeX/references.bib}{Example
      databse}}
\end{frame}

\begin{frame}
  \frametitle{Mathematics}
  Very powerful facilities.  May be enhanced by
  \lstinline{amsmath} packages (best advice is to \emph{always} load
  \lstinline{amsmath}).

  \begin{description}\setlength{\itemsep}{0pt}
  \item[Inline] Here is a formula:
    \begin{math}\sum_{i=1}^{n}i=\frac{n(n+1)}{2}\end{math}; isn't it
    beautiful?
  \item[Displayed] Here is a formula:
    \begin{equation}
      \sum_{i=1}^{n}i=\frac{n(n+1)}{2}
    \end{equation}
    Isn't it beautiful?
  \end{description}
\end{frame}

\subsection{Definitions}

\begin{frame}[fragile]
  \frametitle{Definitions}
  
  A major reason for using \LaTeX{}.  Create special-purpose commands
  and environments for the structures in \emph{your} document.

  To define a command called \lstinline|\UoY| that prints `The
  University of York':\\
  \lstinline|\newcommand*{\UoY}{The University of York}|

  To define a command that has two parameters:\\
  \lstinline|\newcommand*{\C}[2]{_{#1}C^{#2}}|\\
  \lstinline|\begin{math}\C{x+2}{3y}\end{math}| type-sets as
  \textcolor{darkblue}{$_{x+2}C^{3y}$}.
\end{frame}

\begin{frame}[fragile]
  \frametitle{Case study: Cryptographic protocols}
  \framesubtitle{Syntax and form of messages}
  A \emph{message} has three components: \emph{sender},
  \emph{receiver} and \emph{content}.  So we write our document in
  terms of a command \lstinline|\msg| that has 3 parameters:
  \begin{lstlisting}[gobble=4]
    \newcommand*{\msg}[3]{BODY}
  \end{lstlisting}

  Two possible definitions for \lstinline|BODY|:
  \begin{enumerate}
  \item \lstinline|#1\rightarrow#2:#3|
  \item \lstinline|#2\Leftarrow\left[#3\right]\Leftarrow#1|
  \end{enumerate}
  
  The call \lstinline|\msg{S}{R}{C^{A^{B}}}| produces
  \begin{enumerate}
  \item \textcolor{darkblue}{$S\rightarrow R:C^{A^{B}}$} or
  \item\textcolor{darkblue}{$R\Leftarrow\left[C^{A^{B}}\right]\Leftarrow
      S$} respectively. 
  \end{enumerate}
\end{frame}

\begin{frame}[fragile]
  \frametitle{Case study: Cryptographic protocols}
  \framesubtitle{Syntax of message sequences}

    A protocol is a sequence of messages.  So we write our document in
  terms of an environment that collects a sequence of messages.

  We will write, for example:
  \begin{lstlisting}[gobble=4]
    \begin{protocol}
           \msg{A}{B}{X,Y,Z}
      \sep \msg{B}{C}{W,X}
      \sep \msg{C}{B}{W,X'}
    \end{protocol}
  \end{lstlisting}
\end{frame}


\begin{frame}[fragile]
  \frametitle{Case study: Cryptographic protocols}
  \framesubtitle{Desired form of message sequences}

  Now we design the printed form.
  \begin{enumerate}
  \item The output should have numbered messages to which labels can
    be attached.  Each message should be printed on a line of its own.
  \item The definitions of \lstinline|\msg| and \lstinline|\sep|
    should be local to the environment.
  \end{enumerate}
\end{frame}

\begin{frame}[fragile]
  \frametitle{Case study: Cryptographic protocols}
  \framesubtitle{Form of message sequences}
    \begin{lstlisting}[gobble=4]
    \newcounter{msgnumber}
    \newenvironment*{protocol}
    { % set up
      \setcounter{msgnumber}{0}%
      \newcommand*{\msg}[3]{%
        \refstepcounter{msgnumber}%
        \themsgnumber&##1&##2&##3}
      \newcommand*{\sep}{\\}
      \begin{math}\displaystyle%
        \begin{array}%
          {r@{.\quad}l@{\rightarrow}l@{\;:\;}l}}
    { % finalise
      \end{array}\end{math}}
  \end{lstlisting}
\end{frame}

\begin{frame}[fragile]
  \frametitle{Case study: Cryptographic protocols}
  \framesubtitle{The end product}
  \begin{tabular}{l@{\hspace{4em}}l}
    Source&Output\\
    \begin{lstlisting}[gobble=6]
      \begin{protocol}
             \msg{A}{B}{X,Y,Z}
        \sep \msg{B}{C}{W,X}
        \sep \msg{C}{B}{W,X'}
      \end{protocol}
    \end{lstlisting}
    &
    \color{darkblue}
    \newcounter{msgnumber}
    \newenvironment*{protocol}
    {\setcounter{msgnumber}{0}%
      \newcommand*{\msg}[3]{%
        \refstepcounter{msgnumber}\themsgnumber&##1&##2&##3}
      \newcommand*{\sep}{\\}
      \begin{math}\displaystyle%
        \begin{array}{r@{.\quad}l@{\rightarrow}l@{\;:\;}l}}
        {\end{array}
      \end{math}}
    \begin{protocol}
      \msg{A}{B}{X,Y,Z}
      \sep
      \msg{B}{C}{W,X}
      \sep
      \msg{C}{B}{W,X'}
    \end{protocol}
  \end{tabular}
\end{frame}

\section{Running \LaTeXe{}}

\begin{frame}[fragile]
  \frametitle{How to run LaTeX}
  \framesubtitle{The processing cycle}

  \begin{enumerate}
  \item Create \lstinline{<source>.tex}, \lstinline{<source>.ldf},
    bibliographic files, \&c.
  \item Run PDF(E)\LaTeXe{} (Using \TeX{}Live on Departmental Linux:
    `\lstinline{pdflatex <source>}').  Collects auxiliary information
    in \lstinline{<source>.aux}, \lstinline{<source>.toc}, \&c. and
    creates \lstinline{<source>.pdf}.
  \item Run Bib\TeX{} (`\lstinline{bibtex <source>}').  This uses the
    auxiliary information to determine database files and writes
    \lstinline{<source>.bbl} file.
  \item Run PDF(E)\LaTeXe{} (`\lstinline{pdflatex <source>}') a second
    time.  Collects auxiliary information in \lstinline{<source>.aux},
    \lstinline{<source>.toc}, \&c., including bibliographic cross
    references.
  \item Run PDF(E)\LaTeXe{} (`\lstinline{pdflatex <source>}') a third
    time.  There should now be enough auxiliary information to
    generate the final version of \lstinline{<source>.pdf}.
  \end{enumerate}
\end{frame}

\begin{frame}
    \frametitle{How to run LaTeX}
    \framesubtitle{Helpful tools}
    \begin{itemize}
    \item Process can be eased by tools such as
      \begin{itemize}
      \item AUC\TeX{} package for \lstinline{emacs} (any platform).
      \item Mac\TeX{} on Apple
      \item Mik\TeX{} and WinEDT on Microsoft systems.
      \item Eclipse plugin (any platform).
      \end{itemize}
    \item Incremental processing and errors do not mean repeating the
      whole process: for example, Bib\TeX{} only needs to be re-run if
      the bibliographic files change or a new citation is added.
    \item Most tools also give help with managing Bib\TeX{} databases;
      there are also many free-standing tools available.
    \end{itemize}
  \end{frame}
\end{document}

