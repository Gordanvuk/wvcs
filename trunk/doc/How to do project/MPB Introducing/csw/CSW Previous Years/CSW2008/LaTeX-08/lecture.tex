\documentclass{gel}
\talktitle{Introduction to \LaTeX}
\slidetitlecolor{blue}
\pdfauthor{Jeremy Jacob}
\usepackage{amsmath,listings}
\definecolor{darkgreen}{rgb}{0,0.5,0}
\definecolor{darkblue}{rgb}{0,0,0.5}
\lstset{language=[LaTeX]TeX,
  basicstyle={\color{darkgreen}},
  morekeywords={%
    part,chapter,subsection,subsubsection,%
    paragraph,subparagraph,%
    appendix}}

\newcommand*{\guide}[1][]{%
  \href{http://www-course.cs.york.ac.uk/csw/LaTeX/example.pdf}{AGtTSPRiLwtUC}%
}

\begin{document}

\begin{cgel}[Title]{}
  \Huge
  \vspace{-3\baselineskip}
  An Introduction to Type-setting projects in \LaTeX{} with the
  \lstinline|UoYCSProject| class
  \vfill
  \href{http://www-users.cs.york.ac.uk/~jeremy/}{Jeremy Jacob}
  \vfill
  2007 October 31
  \vfill
\end{cgel}

\begin{gel}[What is LaTeX?]{What is \LaTeX?}
  \LARGE

  \LaTeX{} is a \emph{document description language} built on top of
  Donald Knuth's \TeX{} type-setting engine.
  
  \large
 
  Cf.\ \texttt{HTML} and \texttt{SGML}/\texttt{XML} applications.
\end{gel}

\begin{gel}{A minimal document}
  \large

  \begin{tabular}{lcl}
    Source&&Output\\
    \begin{lstlisting}[gobble=6,showspaces=true]
      \documentclass{minimal}
      \begin{document}
      Hello World.
      \end{document}
  \end{lstlisting}
  &&
  \color{darkblue}\rmfamily\tiny
  \framebox[60mm][l]{\begin{tabular}{l}\\Hello World.\\[70mm]\end{tabular}}
\end{tabular}
\end{gel}

\begin{gel}[Why use LaTeX]{Why use \LaTeX{}?}
  \begin{itemize}\setlength{\itemsep}{0pt}
  \item The sophisticated type-setting algorithm of \TeX, and the
    enhanced algorithm of \emph{pdfe(la)tex}.  (See
    \href{http://www.tug.org/texshowcase/}{the \TeX{} showcase}.)
  \item The huge number of pre-defined packages for doing common
    things. (See
    \href{http://www.tex.ac.uk/tex-archive/help/Catalogue/}{the \TeX{}
      catalogue}.)
  \item The ability to define your own special purpose structures.
  \item Stable basis.
  \item Good for large, academic documents.
  \end{itemize}
\end{gel}

\begin{gel}{References}
  \LARGE

  There are many good references for \TeX{} and friends.
  
  See
  ``\href{http://www-course.cs.york.ac.uk/csw/LaTeX/example.pdf}{\emph{A
      guide to type-setting project reports in \LaTeX{} with the
      \emph{UoYCSproject}} class}''.
\end{gel}

\begin{gel}{UoYCSProject --- a class for project reports}
  \large

  There are many pre-defined document classes:

  \begin{description}  \setlength{\itemsep}{0pt}
  \item[Base] minimal, article, report, book, letter, slides.
  \item[KOMA-Script] scrartcl, scrreprt, scrbook, scrlttr2.
  \item[Memorandum] memorandum.
  \item[Others] \ldots, beamer, \ldots,
  \href{http://www-course.cs.york.ac.uk/csw/LaTeX/UoYCSproject.cls}{UoYCSproject},
  \ldots 
  \end{description}
\end{gel}

\begin{gel}{Text, commands and environments}
  A \LaTeX{} source is a mix of:\vspace{-1.5\baselineskip}
  \begin{description}\setlength{\itemsep}{0pt}
  \item[text] \lstinline|Some text.|,
  \item[commands] \lstinline|\LaTeX|, \lstinline|\frac{2}{3}|, and
  \item[environments]\ \\
    \begin{lstlisting}[gobble=6]
      \begin{verse}
        APRIL is the cruellest month, breeding
        \\Lilacs out of the dead land, mixing
        \\...
      \end{verse}
    \end{lstlisting}
  \end{description}
\end{gel}

\begin{gel}[The anatomy of a LaTeX source]{The anatomy of a \LaTeX{}
    source}
  \Large
  \begin{lstlisting}[morekeywords={document},gobble=4]
    \documentclass[class options]{class name}
    preamble (definitions and declarations)
    \begin{document} % this is a comment.
    body 
    \end{document}
  \end{lstlisting}
\end{gel}

\begin{gel}{The anatomy of a UoYCSproject preamble}
 
  {\small
  \begin{lstlisting}[
    gobble=4,
    showspaces=true,
    morekeywords={supervisor,MEng,wordcount,excludes,dedication,abstract,document}]
    \documentclass[citation style]{UoYCSproject}
    % Order of declarations does not matter.
    \author{Anne Student-Name}
    \title{A Solution to the Problem of $\mathit{P}=\mathit{NP}$}
    \date{30 February 2000}
    \supervisor{Prof. Z. Soporific}
    \MEng
    \wordcount{2,345}\excludes{Appendix~\ref{sec:code}}
    \dedication{To My Cat, Jeoffery}
    \abstract{The well known problem of $P=NP$ is explained, together
      with its significance and a brief history of attempts to solve
      it.  An ingenious solution is presented.}
    \begin{document}
    ...
    \end{document}
  \end{lstlisting}
  }

  A full list of declarations is given in \guide[, Figure~7.1, P~46].
\end{gel}

\begin{gel}{Extra definitions and package loading}
  \large

  You can load extra packages and make your own definitions.
  
  These go in a file with the same name as your main file, but
  extension `ldf'. \textbf{\textcolor{red}{\textsc{This is different
        to the way all other classes work.}}}  (I have implemented
  UoYCSproject in this way to ensure that packages are loaded in the
  correct order.)
  
  Useful packages include: \lstinline|listings|, \lstinline|graphics|,
  \lstinline|graphicx|, \lstinline|pgf/tikz|, \lstinline|amsmath|.
\end{gel}

\begin{gel}{The anatomy of the body}
  \Large
  \begin{description}\setlength{\itemsep}{0pt}
  \item[Front matter] Title pages, abstract, contents, \emph{\&c}.
  \item[Main matter] The text, divided into (parts,) chapters (,
    sections, subsections, subsubsections, paragraphs and
    subparagraphs).
  \item[Back matter] Bibliography, appendices \emph{\&c}.
  \end{description}
\end{gel}

\begin{gel}{Front matter}
  \large
  \begin{lstlisting}[gobble=4]
    \maketitle % Compulsory: title pages, table of contents
    \listoffigures % Optional: the list of figures
    \listoftables % Optional: the list of tables
    ... % Optional, package dependent lists,
    ....% e.g. \lstlistoflistings
  \end{lstlisting}
\end{gel}

\begin{gel}{Main matter}
  \large
  \begin{lstlisting}[gobble=4]
    \part{title}          % Optional
    \chapter{title}       % Compulsory
    \section{title}       % Optional
    \subsection{title}    % Optional
    \subsubsection{title} % Optional
    \paragraph{title}     % Optional
    \subparagraph{title}  % Optional
    Text.  Text.
  \end{lstlisting}
\end{gel}

\begin{gel}{Back matter}
  \large
  \begin{lstlisting}[gobble=4]
    \bibliography{file1,file2} % Construct bibliography
    \appendix % remaining chapters are appendices
    \chapter{title}       % One per appendix
    \section{title}       % Optional
    \subsection{title}    % Optional
    \subsubsection{title} % Optional
    \paragraph{title}     % Optional
    \subparagraph{title}  % Optional
    Text.  Text.
  \end{lstlisting}
\end{gel}

\begin{gel}{Text elements}
  \large
  \begin{description}\setlength{\itemsep}{0pt}
  \item[Characters] Can control series, family, shape, colour and size
  of each text character.  See \guide[, \S6.3.3].
  \item[Sentences] \lstinline[showspaces=true]|Sentence one.  Sentence two.|
  \item[Paragraphs]\ \\[-\baselineskip]
    \begin{lstlisting}[showspaces=true,gobble=6]
      Paragraph one.  %  blank line separates paragraphs

      Paragraph two.
    \end{lstlisting}
  \end{description}
\end{gel}

\begin{gel}{Special features}
  \Large
  \begin{description}\setlength{\itemsep}{0pt}\setlength{\parskip}{0pt}
  \item[Context dependent emphasis]
  \item[Cross references] Sectional units, floats, equations, \emph{\&c.}
  \item[Quotations] Short and long
  \item[Lists] Bulleted, numbered and labelled
  \item[Tables]
  \item[Pictures]
  \item[Floats] Tables, Figures and others.
  \end{description}
\end{gel}

\begin{gel}{Citations and the bibliography}
Through the \emph{natbib} package.  Two style options:
\vspace{-\baselineskip}
\begin{description}
\item[Harvard] Use option: \lstinline{authoryearcitations}
  \vspace{-\baselineskip}
  \begin{description}\setlength{\itemsep}{0pt}\setlength{\parskip}{0pt}
  \item[Cite as text] `As \textcolor{red}{Joyce (1939)} says, life
    is complex.'
  \item[Cite in parenthesis] `Life is complex
    \textcolor{red}{(Joyce, 1939)}.'
  \end{description}
  \vspace{-2\baselineskip}
\item[Toronto] Use option: \lstinline{numericalcitations}
  \vspace{-\baselineskip}
  \begin{description}\setlength{\itemsep}{0pt}\setlength{\parskip}{0pt}
  \item[Cite as number] `As Joyce \textcolor{red}{[17]} says, life
    is complex.' or `Life is complex \textcolor{red}{[17]}.'
  \end{description}
  \vspace{-\baselineskip}
\end{description}

Citations are kept in a
\href{http://www-course.cs.york.ac.uk/csw/LaTeX/references.bib}{database
  in a flat file} and processed by a program called Bib\TeX{} before
inclusion in output file.
\end{gel}

\begin{gel}{Mathematics}
  \large Very powerful facilities.  May be enhanced by
  \lstinline{amsmath} packages (best advice is to \emph{always} load
  \lstinline{amsmath}).

  \begin{description}\setlength{\itemsep}{0pt}
  \item[Inline] Here is a formula:
    \begin{math}\sum_{i=1}^{n}i=\frac{n(n+1)}{2}\end{math}; isn't it
    beautiful?
  \item[Displayed] Here is a formula:
    \begin{equation}
      \sum_{i=1}^{n}i=\frac{n(n+1)}{2}
    \end{equation}
    Isn't it beautiful?
  \end{description}
\end{gel}

\begin{gel}{Definitions}
  \large
  A major reason for using \LaTeX{}.  Create special-purpose commands
  and environments for the structures in \emph{your} document.

  To define a command called \lstinline|\UoY| that prints `The
  University of York':\\
  \lstinline|\newcommand*{\UoY}{The University of York}|

  To define a command that has two parameters:\\
  \lstinline|\newcommand*{\C}[2]{_{#1}C^{#2}}|\\
  \lstinline|\begin{math}\C{x+2}{3y}\end{math}| type-sets as
  \textcolor{darkblue}{$_{x+2}C^{3y}$}.
\end{gel}

\begin{gel}{Case study: cryptographic protocols --- 1}
  \large
  A \emph{message} has three components: \emph{sender},
  \emph{receiver} and \emph{content}.  So we write our document in
  terms of a command \lstinline|\msg| that has 3 parameters.

  Two possible definitions:\\
  \lstinline|\newcommand*{\msg}[3]{#1\rightarrow#2:#3}|\\
  \lstinline|\newcommand*{\msg}[3]{#2\Leftarrow\left[#3\right]\Leftarrow#1}|\\
  \lstinline|\msg{S}{R}{C^{A^{B}}}| produces either
  \textcolor{darkblue}{$S\rightarrow R:C^{A^{B}}$} or
  \textcolor{darkblue}{$R\Leftarrow\left[C^{A^{B}}\right]\Leftarrow
    S$}.
\end{gel}

\begin{gel}{Case study: cryptographic protocols --- 2a}
  \large
  A protocol is a sequence of messages.  So we write our document in
  terms of an environment that collects a sequence of messages.

  We will write, for example:
  \begin{lstlisting}[gobble=4]
    \begin{protocol}
           \msg{A}{B}{X,Y,Z}
      \sep \msg{B}{C}{W,X}
      \sep \msg{C}{B}{W,X'}
    \end{protocol}
  \end{lstlisting}
\end{gel}

\begin{gel}{Case study: cryptographic protocols --- 2b}
  \Large
  Now we design the printed form.

  The output should have numbered messages to which labels can be
  attached.  Each message should be printed on a line of its own.
  
  The definitions of \lstinline|\msg| and \lstinline|\sep| should be
  local to the environment.
\end{gel}

\begin{gel}{Case study: cryptographic protocols --- 2c}
  \large
  \begin{lstlisting}[gobble=4]
    \newcounter{msgnumber}
    \newenvironment*{protocol}
     {\setcounter{msgnumber}{0}%
      \newcommand*{\msg}[3]{%
        \refstepcounter{msgnumber}\themsgnumber&##1&##2&##3}
      \newcommand*{\sep}{\\}
      \begin{math}\displaystyle%
        \begin{array}{r@{.\quad}l@{\rightarrow}l@{\;:\;}l}}
     {\end{array}\end{math}}
  \end{lstlisting}
\end{gel}

\begin{gel}{Case study: cryptographic protocols --- 2d}
  \large
  \begin{tabular}{l@{\hspace{4em}}l}
    Source&Output\\
    \begin{lstlisting}[gobble=6]
      \begin{protocol}
             \msg{A}{B}{X,Y,Z}
        \sep \msg{B}{C}{W,X}
        \sep \msg{C}{B}{W,X'}
      \end{protocol}
    \end{lstlisting}
    &
    \color{darkblue}
    \newcounter{msgnumber}
    \newenvironment*{protocol}
    {\setcounter{msgnumber}{0}%
      \newcommand*{\msg}[3]{%
        \refstepcounter{msgnumber}\themsgnumber&##1&##2&##3}
      \newcommand*{\sep}{\\}
      \begin{math}\displaystyle%
        \begin{array}{r@{.\quad}l@{\rightarrow}l@{\;:\;}l}}
        {\end{array}
      \end{math}}
    \begin{protocol}
      \msg{A}{B}{X,Y,Z}
      \sep
      \msg{B}{C}{W,X}
      \sep
      \msg{C}{B}{W,X'}
    \end{protocol}
  \end{tabular}
\end{gel}

\begin{gel}[How to run LaTeX - 1]{How to run \LaTeX{} --- 1}
  \begin{enumerate}\setlength{\itemsep}{0pt}
  \item Create \texttt{<source>.tex}, \texttt{<source>.ldf},
    bibliographic files, \emph{\&c.}
  \item Run PDF(E)\LaTeXe{} (Using \TeX{}Live on Departmental Linux:
    `\texttt{pdflatex~<source>}').  Collects auxiliary information in
    \texttt{<source>.aux}, \texttt{<source>.toc}, \emph{\&c.} and
    creates \texttt{<source>.pdf}.
  \item Run Bib\TeX{} (`\texttt{bibtex~<source>}').  This uses the
    auxiliary information to determine database files and writes
    \texttt{<source>.bbl} file.
  \item Run PDF(E)\LaTeXe{} (`\texttt{pdflatex~<source>}').  Collects
    auxiliary information in\\ \texttt{<source>.aux},
    \texttt{<source>.toc}, \emph{\&c.}, including bibliographic cross
    references.
  \item Run PDF(E)\LaTeXe{} (`\texttt{pdflatex~<source>}').  There
    should now be enough auxiliary information to generate the final
    version of \texttt{<source>.pdf}.
  \end{enumerate}
\end{gel}

\begin{gel}[How to run LaTeX - 2]{How to run \LaTeX{} --- 2}
  Process can be eased by tools such as
  \begin{itemize}
  \item \texttt{AUCTeX} package for \texttt{emacs} (Linux and Microsoft).
  \item Mik\TeX{} and WinEDT on Microsoft systems.
  \end{itemize}
  
  Incremental processing and errors do not mean repeating the whole
  process: for example, Bib\TeX{} only needs to be re-run if the
  bibliographic files change or a new citation is added.
\end{gel}
\end{document}
