\documentclass[12pt,14paper]{scrartcl}
\usepackage{listings,UoYstyle}
\lstset{language=[latex]tex,basicstyle=\sffamily,gobble=4}
\title{\LaTeXe\ Problems for CSW}
\author{Jeremy Jacob}
\date{2009 May 28}
\begin{document}
\maketitle

\section{Problems}

These exercises are designed to be run using the \TeX{}live
distribution.  It is best to use \lstinline{pdflatex} (rather than
\lstinline{latex}), and view the output using a PDF viewer.  When
completing these exercises you may wish to have by you a \LaTeX\ book,
or other guide (see the reference manual for
\lstinline{UoYCSproject}).

\subsection{Very basic usage of \LaTeXe}

Jim Hefferon's exercises at
\url{http://tug.ctan.org/tex-archive/info/first-latex-doc/first-latex-doc.pdf}
take you through the construction of a very simple \LaTeXe\ document.

Notes:
\begin{itemize}
\item If you do these exercises on the Department's system you can
  skip the `Get the software' section!
\item A similar `boilerplate' ---but without the \lstinline{ams*}
  packages--- to Hefferon's is provided for you by
  \lstinline{UoYCSproject.cls}, so you do not need to use it for
  documents of that class.
\item I recommend that in your project report you use Bib\TeX, and do
  \emph{not} build a bibliography by hand.
\end{itemize}

\subsection{Simple macro programming}\label{sec:simplemacro}

These exercises introduce you to programming using \TeX\ macros
through the \LaTeXe\ interface.  Suggested solutions can be found in
\autoref{sec:solutions}.

\begin{enumerate}
\item Create an A4 document of the class \lstinline{article} whose
  title is `Greetings', that has you as the author and today's date.
  The body should have two sections, titled `Welcome' and `Farewell',
  in that order.  Each section should have two sentences.  The first
  should be one of `Hello world' or `Goodbye world' as appropriate.
  The second should be a sentence of the shape `See also Section~n',
  where `n' is the number of the other section.  The numbers should
  not appear in your source [hint: use symbolic labels and
  references].
\item Create an identical document, but make it of class
  \lstinline{scrartcl}.  Do you notice a difference in the output?
  Which do you prefer?
\item In the document you have created so far, you should have typed
  `See also' twice.  Introduce a macro, \lstinline{\sa}, to typeset
  `See also', and use it in both places (do not forget to protect the
  following space).
\item In the document you have created so far, you should have typed
  the structure `\textless salute\textgreater\ world.' twice, once for
  salutation `Hello' and once for salutation `Goodbye'.  Introduce a
  macro, \lstinline{\sw}, with one parameter that is the salutation,
  and use it in both places.
\item If the package \lstinline{hyperref} is loaded\footnote{As
    \lstinline{UoYCSproject} does.} then, as well as automatically
  generating bookmarks and internal links, the macro
  \lstinline|\autoref{label}| is available.  This macro knows the kind
  of reference that a label refers to, so you do not have to type
  `\lstinline|Section~|', and so on.  Make your document load the
  package, and modify your document to use \lstinline{\autoref} rather
  than \lstinline{\ref}.
\item Your document consists of one structure repeated twice.  The
  structure has three variables.  In the first instance it is
  instantiated with \lstinline{Welcome}, \lstinline{Hello} and
  \lstinline{Farewell}; in the second it is instantiated with
  \lstinline{Farewell}, \lstinline{Goodbye} and \lstinline{Welcome}.
  Create a 3-parameter macro, \lstinline{\st}, that captures this
  structure and use it.  The body of your document should be
  \lstinline{\maketitle} followed by two calls to your new macro, and
  nothing else.\footnote{If you wish you can add a package to change
    fonts from the default set; to create a good PDF document this
    should always be done.  Suitable packages are \lstinline{mathpazo}
    (Palatino) and \lstinline{mathptmx} (Times Roman), both combined
    with \lstinline{helvet,courier}.  The package
    \lstinline{microtype} is also recommended.  The package
    \lstinline{hyperref} automatically turns references into
    hyperlinks; it must usually be the last loaded package.}
\end{enumerate}

\newpage
\appendix

\section{Possible solutions for \autoref{sec:simplemacro}}
\label{sec:solutions}

\begin{enumerate}
\item Your answer should have a different author, of course!  It may
  also have different labels.  It should have more white space, but I
  have compressed things here, to save paper.
  \begin{lstlisting}
    \documentclass[a4paper]{article}
    \title{Greetings}
    \author{Jeremy Jacob}
    \begin{document}
    \maketitle
    \section{Welcome}\label{sec:Welcome}
    Hello world.  See also Section~\ref{sec:Farewell}.
    \section{Farewell}\label{sec:Farewell}
    Goodbye world.  See also Section~\ref{sec:Welcome}.
    \end{document}
  \end{lstlisting}
\item For completeness:
  \begin{lstlisting}
    \documentclass[a4paper]{scrartcl}
    \title{Greetings}
    \author{Jeremy Jacob}
    \begin{document}
    \maketitle
    \section{Welcome}\label{sec:Welcome}
    Hello world.  See also Section~\ref{sec:Farewell}.
    \section{Farewell}\label{sec:Farewell}
    Goodbye world.  See also Section~\ref{sec:Welcome}.
    \end{document}
  \end{lstlisting}
\item\
  \begin{lstlisting}
    \documentclass[a4paper]{article}
    \title{Greetings}
    \author{Jeremy Jacob}
    \newcommand*{\sa}{See also}
    \maketitle
    \begin{document}
    \section{Welcome}\label{sec:Welcome}
    Hello world.  \sa\ Section~\ref{sec:Farewell}.
    \section{Farewell}\label{sec:Farewell}
    Goodbye world.  \sa\ Section~\ref{sec:Welcome}.
    \end{document}    
  \end{lstlisting}
  \newpage
\item\
  \begin{lstlisting}
    \documentclass[a4paper]{article}
    \title{Greetings}
    \author{Jeremy Jacob}
    \newcommand*{\sa}{See also}
    \newcommand*{\sw}[1]{#1 world.}
    \begin{document}
    \maketitle
    \section{Welcome}\label{sec:Welcome}
    \sw{Hello}  \sa\ Section~\ref{sec:Farewell}.
    \section{Farewell}\label{sec:Farewell}
    \sw{Goodbye}  \sa\ Section~\ref{sec:Welcome}.
    \end{document}    
  \end{lstlisting}
\item\
  \begin{lstlisting}
    \documentclass[a4paper]{article}
    \usepackage{hyperref}
    \title{Greetings}
    \author{Jeremy Jacob}
    \newcommand*{\sa}{See also}
    \newcommand*{\sw}[1]{#1 world.}
    \begin{document}
    \maketitle
    \section{Welcome}\label{sec:Welcome}
    \sw{Hello}  \sa\ \autoref{sec:Farewell}.
    \section{Farewell}\label{sec:Farewell}
    \sw{Goodbye}  \sa\ \autoref{sec:Welcome}.
    \end{document}    
  \end{lstlisting}
  \newpage
\item\
  \begin{lstlisting}
    \documentclass[a4paper]{article}
    \usepackage{hyperref}
    \title{Greetings}
    \author{Jeremy Jacob}
    \newcommand*{\sa}{See also}
    \newcommand*{\sw}[1]{#1 world.}
    \newcommand*{\st}[3]{%
      \section{#1}\label{sec:#1}\sw{#2}  \sa\ \autoref{sec:#3}.%
    }
    \begin{document}
    \maketitle
    \st{Welcome}{Hello}{Farewell}
    \st{Farewell}{Goodbye}{Welcome}
    \end{document}    
  \end{lstlisting}
\end{enumerate}

\end{document}
